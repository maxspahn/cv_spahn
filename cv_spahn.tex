\documentclass[english]{tccv}
\usepackage{babel}

\usepackage[utf8]{inputenc}
\usepackage[T1]{fontenc}


\newcommand{\multilang}[2]{
	\iflanguage{german}{#1}{}
	\iflanguage{ngerman}{#1}{}
	\iflanguage{english}{#2}{}
}



\begin{document}

\part{Max Spahn}{\multilang{Suchend nach einem herausfordernden Praktikum in der IT-Beratung}
	{Searching for challenging internship in IT-Consulting}}

\personal
    [www.linkedin.com/in/max-spahn]
    {Rudolstraße 40\newline 52070 -- Aachen, \multilang{Deutschland}{Germany}}
    {+49 1578 8221336}
    {max.spahn.pro@gmail.com}
	 {https://github.com/maxspahn}

\section{\multilang{Ausbildung}{Education}}

\begin{eventlist}

\item{2018 -- 2019}{RWTH Aachen}
	{\multilang{Allgemeiner Maschinenbau M.Sc.}{General Mech. Engineering M.Sc.}}

	\multilang{
		Abschluss im Oktober 2019\\
		Stipendium : Deutschland-Stipendium}
	{
		Graduation in October 2019\\
		Scholarship : Deutschland-Stipedium}

\item{2013 -- 2018}{RWTH Aachen}
	{\multilang{Maschinenbau B.Sc.}{Mechanical Engineering B.Sc.}}

	\multilang{
		\textbf{Note : 1.8}\\
		Stipendium : Deutschland-Stipendium\\
		Dean's List (unter den besten 5\% im Studiengang)}
	{
		\textbf{Score : 1.8} (best 1.0, worst 5.0)\\
		Scholarship : Deutschland-Stipendium\\
		Dean's List (among the best 5\% in the program)}

\item{2015 -- 2017}{Ecole Centrale Supelec}
	{\multilang{Diplom Allgemeines Ingenieurweses}{Diploma in General Engineering}}

	\textbf{GPA : 3.52/4.0}
	\multilang{
		Abschluss im Okotober 2019\\
		Doppel-Diplom Programm mit der RWTH\\
		Spezialisierung auf Software-Engineerung\\
		Stipendium : Deutsch-Französische Hochschule}
	{
		Graduation in Octobore 2019\\
		Doubl-Degree Program with RWTH\\
		Focus on Software-Engineerung\\
		Scholorship : Deutsch-Französische Hochschule}

\item{2006 -- 2013}{Norbert-Gynmasium \multilang{}{(High-School)}}
	{\multilang{Deutsches Abitur}{German Abitur}}

	\multilang{
		\textbf{Abitur-Note : 1.0}
		Montreal, Kanada : 2010 -- 2011}
	{
		\textbf{Score : 1.0} (best 1.0, worst 5.0)
		Montreal, Canada : 2010 -- 2011}
		
\end{eventlist}

\hrule
\vspace{5pt}
\multilang{
	\textbf{Wichtige Studieninhalte :} Höhere Mathematik I/II/III, Theromdynamik, Mechanik I/II/III, 
	Maschinengestaltung I/II/III, Wärme- und Stoffübertragung, Software-Engineering, Random-Modeling,
	Makroökonomie, Parallel Computing, Simulations Technik}
{
	\textbf{Relevant Coursework :} Mathematics I/II/III, Thermodynics, Mechanics I/II/III, 
	Mechanical Design I/II/III, Heat Transfer, Software-Engineering, Random-Modeling, 
	Economics, Parallel Computing, Simulation Science}   


\section{\multilang{Sprachkenntnisse}{Communication Skills}}

\multilang{
	\begin{factlist}
		\item{Deutsch}{Muttersprache}
		\item{Englisch}{Fließend}
		\item{Französisch}{Flißend}
		\item{Latein}{Großes Latinum}
	\end{factlist}
}
{
	\begin{factlist}
		\item{German}{Native Speaker}
		\item{English}{Fluent}
		\item{French}{Fluent}
		\item{Latin}{5 years in school}
	\end{factlist}
}

\section{\multilang{Berufserfahrung}{Work experience}}

\begin{eventlist}

\item{\multilang{Juli 2017 -- Dezember 2017}{July 2017 -- December 2017}}
	{\multilang{KUKA Roboter AG.}{KUKA Robots AG.}}
	{\multilang{Praktikum, Konzern-Forschung}{Internship in Corporate Research}}
	
	\multilang{
		Entwicklung einer Software-Toolbox zur Simulation und Animation industrieller Roboter:		
		Implementierung in \textsc{Matlab} mit Integration von \textsc{c++} für die 
		rechenintensiven Funktionen. Entwicklung und Implementierung von Algorithmen für die 
		analytische Lösung der \textit{Inversen Kinematik} und \textit{Bahnplanung}.}
	{
		Development of a Software-Toolbox for the simulation and animation of industrial 
		robots: Implementation in \textsc{Matlab} with integration of \textsc{c++} for the 
		computational costly functions. Design and implementation of algorithms for 
		\textit{Inverse Kiinematics} and \textit{Trajectory Planning}.}

\item{\multilang{Juli 2016 -- September 2016}{July 2016 -- September 2016}}
	{\multilang{CATS, Lehrstuhl der RWTH}{CATS, Institute RWTH}}
	{\multilang{HiWi-Beschäftigung}{Student Job}}

	\multilang{
		HiWi am Lehrstuhl für computergestützte Analyse technischer Systeme: Entwicklung einer GUI
		in \textsc{Python} für eine 3D-Modell Datenbank. Implementierung von Wrappern, sowohl 
		für \textsc{c++} als auch für \textsc{FORTRAN}.}
	{
		Student Job at the Chair for Computational Analysis of Technical Systems: Development of a 
		Graphical User Interface in \textsc{Python} for a 3D-Model library. Implementation of 
		wrapper-classes in \textsc{c++} and \textsc{FORTRAN}.}

\item{\multilang{Juli 2014 -- September 2015}{July 2014 -- September 2015}}
	{\multilang{WZL, Lehrstuhl der RWTH}{WZL, Institute RWTH}}
	{\multilang{HiWi-Beschäftigung}{Student Job}}

	\multilang{
		HiWi am Werkzeugmaschinenlabor der RWTH als Assistent eines Promotionsstudenten: Entwicklung
		eines Simulations-Tool zur Optimierung des Anlauf-Managements. Teil der Abteilung, die sich 
		mit der Entwicklung neuer Technologien, Technologien und deren Evaluierung beschäftigt.}
{
		Assistant to scientific research for PhD student: Development of a simulation tool to
		optimize the Ramp-Up Management. Part of a team developing new technologise, technology chains
		and their evaluation.}

\end{eventlist}


\section{\multilang{Projeket}{Projects}}

\begin{eventlist}

\item{\multilang{Januar 2018 -- März 2018}{January 2018 -- March 2018}}
	{\multilang{KUKA Roboter AG.}{KUKA Robots AG.}}
	{\multilang{Bachelor Arbeit}{Bachelor Thesis}}

	\multilang{
		Entwicklung eines robusten und vorhersehbaren Algorithmus zur numerischen Lösung der 
		Inversen Kinematik serieller Roboter unter Verwendung des 
		\textit{Gradienten-Abstiegs} Verfahren
		und des \textit{Newton-Raphson} Verfahren als zentrale Methoden. Implementierung in 
		\textsc{Matlab}.}
	{
		Development of a robust and predictable algorithm for the numerical solution of the 
		inverse kinematic problem of serial robots using \textit{Gradient-Descent} and 
		\textit{Newton-Raphson} as the key methods. Implemented in \textsc{Matlab}.}

\end{eventlist}


\begin{yearlist}
	\item{2017}{Kpelo}
		{	
	\multilang{Automatische Textkorrektur implementiert in \textsc{Python}}
		{\textsc{Python}-based Spellchecker using Machine Learning}
	}

	\item{2017}{MyFoodora}
	{	
	\multilang{Lieferservice Software implementiert in \textsc{Java}}
		{Food-Delivery software implented in \textsc{Java}},
		 \href{https://github.com/maxspahn/MyFoodora}{MyFoodora on Git-Hub}
	 }

\end{yearlist}

\section{\multilang{Software Kenntnisse}{Software Skills}}

\begin{factlist}
	\item{\multilang{Fortgeschritten}{Advanced}}
		{Matlab, Python, git, Linux, LaTeX}
	\item{\multilang{Gutes Niveau}{Intermediate}}
		{Java, c++, Microsoft Office}
	\item{\multilang{Grundkenntisse}{Basic Level}}
		{Postgres, FORTRAN}
\end{factlist}

\end{document}
